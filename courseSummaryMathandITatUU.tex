\documentclass{article}

\usepackage{longtable}

\title{A summary of courses at Uppsala University}

\author{Johannes Graner}

\begin{document}
  
  \maketitle

  \section{Courses taken}

  \subsection{Undergraduate (kandidatprogram)}

  \begin{longtable}{|c|c|c|}
    \hline
    Year & Period & Courses (credits, field, advanced) \\
    \hline
    1 & 1 & Linear Algebra and Geometry I (5) \\
      &   & Introduction to Studies in Mathematics (5) \\
      &   & Single Variable Calculus M (5/10) \\
      &   & Honours Course in Mathematics (2.5/5) \\
    \hline
      & 2 & Scientific Computing I (5, Computer Science) \\
      &   & Algebra I (5) \\
      &   & Single Variable Calculus M (5/10) \\
      &   & Honours Course in Mathematics (2.5/5) \\
    \hline
      & 3 & Linear Algrebra II (5) \\
      &   & Computer Programming I (5, Computer Science) \\
      &   & Several Variable Calculus M (5/10) \\
      &   & Special Course in Mathematics II (2.5/5) \\
    \hline
      & 4 & Logic and Proof Techniques I (5) \\
      &   & Algebra II (5) \\
      &   & Several Variable Calculus M (5/10) \\
      &   & Special Course in Mathematics II (2.5/5) \\
      &   & Affine and Projective Geometry (5) \\
    \hline
    2 & 1 & Ordinary Differential Equations I (5) \\
      &   & Probability Theory I (5) \\
      &   & Computer Programming II (5, Computer Science) \\
      &   & Real Analysis (5/10) \\
    \hline
      & 2 & Fourier Analysis (5) \\
      &   & Inference Theory I (5) \\
      &   & Real Analysis (5/10) \\
    \hline
      & 3 & Linear Algebra III (5) \\
      &   & Stochastics (5) \\
      &   & Scientific Computing II (5, Computer Science) \\
      &   & Complex Analysis (5/10) \\
    \hline
      & 4 & Basic Topology (5) \\
      &   & Regression Analysis (5) \\
      &   & Complex Analysis (5/10) \\
    \hline
    3 & 1 & Probability Theory II (5) \\
      &   & Functional Programming I (5, Computer Science, advanced) \\
      &   & Differential Geometry (5/10) \\
      &   & Multivariate Methods (5/10) \\
    \hline
      & 2 & Inference Theory II (5) \\
      &   & Differential Geometry (5/10) \\
      &   & Multivariate Methods (5/10) \\
    \hline
      & 3 & Scientific Computing III (5, Computer Science, advanced) \\
      &   & Computer Intensive Statistics and Data Mining (5/10, advanced) \\
      &   & Degree Project C in Mathematics (7.5/15) \\
    \hline
      & 4 & Computer Intensive Statistics and Data Mining (5/10, advanced) \\
      &   & Degree Project C in Mathematics (7.5/15) \\
    \hline
  \end{longtable}

  \subsection{Master's programme}

  All courses in the following table are on advanced level.

  \begin{longtable}{|c|c|c|}
    \hline
    Year & Period & Courses (credits, field) \\
    \hline
    1 & 1 & Introduction to Data Science (5/10, Data Science) \\
      &   & Theoretical Statistics (5/10) \\
      &   & Integration Theory (5/10) \\
    \hline
      & 2 & Introduction to Data Science (5/10, Data Science) \\
      &   & Theoretical Statistics (5/10) \\
      &   & Integration Theory (5/10) \\
      &   & Generalised Linear Models (5) \\
    \hline
      & 3 & Statistical Machine Learning (5, Data Science) \\
      &   & Markov Processes (10) \\
      &   & Bayesian Statistics (5/10) \\
    \hline
      & 4 & High Performance and Parallel Computing (7.5, Computer Science) \\
      &   & Analysis of Time Series (10) \\
      &   & Bayesian Statistics (5/10) \\
    \hline
    2 & 1 & Advanced Probabalistic Machine Learning (7.5, Data Science) \\
      &   & Accelerator-Based Programming (7.5, Computer Science) \\
      &   & Data Mining (7.5, Data Science) \\
    \hline
      & 2 & Analysis of Categorical Data (5) \\
      &   & Database Design II (5, Computer Science) \\
      &   & Scientific Visualization (5, Computer Science) \\
    \hline
      & 3 & Degree Project E in Mathematics (15/30) \\
    \hline
      & 4 & Degree Project E in Mathematics (15/30) \\
    \hline
  \end{longtable}

  \section{Reflections}

  For a mathematics student who wants more computer science and software engineering, I would recommend the following courses (roughly in this order).

  \begin{enumerate}
    \item Computer Programming I and II for the basics of programming.
    \item Scientific Computing I, II, and III for converting mathematical formulations into code.
    \item High-Performance and Parallel Computing for understanding how computers works 'under the hood' and improving performance.
    \item Functional Programming for a more mathematical approach to programming. Not as related to software engineering, but broadens the view on what programming can be.
    \item Accelerator-Based Programming for GPU programming. Mostly of interest in high-performance computing, and to a lesser extent machine learning engineering.
  \end{enumerate}

  In addition to these courses, a course focused on the practical side of software development would be very welcome.
  Such a course would introduce concepts and workflows such as containerization (Docker), packaging of software, version control (git/GitHub), basic terminal commands, etc.
  This course would preferably be given before High-Performance and Parallel Computing.

  For a Master's programme, Computer Programming I and II, and Scientific Computing I and II, should be prerequisites.

  The course High-Performance and Parallel Computing currently exists in two versions, a 7.5 and a 10 credit version.
  For the students of a Technical Mathematics programme, the 10 credit version should be strongly considered since the language in the course is C and the 10 credit course includes an introduction to C (which I did not know when applying).

  I would also have liked to take the course Parallel and Distributed Programming, but unfortunately I was not able to fit it into my study schedule.
  This would be of interest to students highly interested in high-performance computing.

  \subsection{Note on study pace}
  Most of the time, I have had a higher than 100\% study pace.
  The courses sum up to 355 credits, which makes the average study pace close to 120\%.
  However, this is not evenly spread out, and the study pace in individual periods range from 100\% to 150\%.

\end{document}